\clearpage
\section{CONCLUSION}
\
%%%%%%%%%%%%%%%%%%%%%%%%%%%%%%%%%%%%%%%%%%%%%%%%%%%%%%%%%%%%%%%%%%%%%%%%%%%%%%%%%%%%%%%%
%%%%%%%%%%%%%%%%%%%%%%%%%%%%%%%%%%%%%%%%%%%%%%%%%%%%%%%%%%%%%%%%%%%%%%%%%%%%%%%%%%%%%%%%

Methods of computational chemistry were used to study the structural and thermodynamical properties of one representative biocompatible polymer (PLA) and four selected APIs. At fisrt, all studied materials were studied separately as neat substances. Then the properties of amorphous mixtures were determined with regard to interactions of their components at the microscale to establish an interpretation of their macroscopic behavior. 

The glass transition temperature of polylactic acid was determined using molecular dynamics methods and validated by comparison with experimental data. Same validation was done also for densities. Then, binary mixtures of the four selected API with polylactic acid were studied focusing on their intermolecular interactions, mainly the potential hydrogen bonding. Mean-squared displacement and radial distribution functions of the mixtures and neat API were discussed. Series of molecular-dynamics simulations were performed in order to get the glass-transition temperature of the mixtures.
 
 je treba shrnou tredy spoctenych Tg, provazat je s RSD a MSD
 
 vse v podobe nekolika vet, aby to bylo srozumitelne pro ctenare laika
 
 v principu se pro 3 API ze 4 (IBU, NAP, INDO) ukazalo, ze interaguji s PLA dost silne na to, aby Tg smesi vzrostlo oproti cistym latkam
 
 to co jsme pozorovali pro INDO pri 500 K, ze ma smes vetsi MSD nez cista API bude nejaky artefakt toho, ze PLA se nam v nasi MD prilis konformacne rozhybe a to pak zresluje MSD i smesi. Pri 300 K se nic tak anomalniho pro INDO ani jeho smesi nedeje a stabilizujici INDO-PLA interakce tam jsou
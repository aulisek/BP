\clearpage
\section{CONCLUSION}
\
%%%%%%%%%%%%%%%%%%%%%%%%%%%%%%%%%%%%%%%%%%%%%%%%%%%%%%%%%%%%%%%%%%%%%%%%%%%%%%%%%%%%%%%%
%%%%%%%%%%%%%%%%%%%%%%%%%%%%%%%%%%%%%%%%%%%%%%%%%%%%%%%%%%%%%%%%%%%%%%%%%%%%%%%%%%%%%%%%

Methods of computational chemistry were used to study the structural and thermodynamical properties of one representative biocompatible polymer (PLA) and four selected APIs. At fisrt, all studied materials were studied separately as neat substances. Then the properties of amorphous mixtures were determined with regard to interactions of their components at the microscale to establish an interpretation of their macroscopic behavior. 

The glass transition temperature of polylactic acid was determined using molecular dynamics methods and validated by comparison with experimental data. Same validation was done also for densities. Then, binary mixtures of the four selected API with polylactic acid were studied focusing on their intermolecular interactions, mainly the potential hydrogen bonding. Mean-squared displacement and radial distribution functions of the mixtures and neat API were discussed. Series of molecular-dynamics simulations were performed in order to get the glass-transition temperature of the mixtures.

We found that the glass transition temperature for binary mixtures for 3 out of the 4 selected active pharmaceutical ingredients (ibuprofen, indomethacine and naproxen) increased compared to the values for the neat substances. This is consistent with the observation that these APIs exhibited sufficiently strong interactions with the polymer excipient. In contrast, in the case of carbamazepine, the interactions were not sufficient to stabilize the API with the polymer and the mixture did not show an increase in the glass transition temperature. 
